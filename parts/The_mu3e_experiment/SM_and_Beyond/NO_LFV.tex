Neutrino Oscillation was first proposed in 1958 by B.Pontecorvo~\cite{berger2025particle}. 
\textcolor{red}{Inspired by the phenomenon of neutral kaon oscillation, he suggested that if neutrinos had a non-zero mass, they could change their flavour while propagating through space.}
\\
The existance of neutrino oscillation was experimentally confirmed in 1998 by the Super-Kamiokande experiment in Japan, which observed the oscillation of atmospheric neutrinos. 
This discovery provided strong evidence that neutrinos have mass and that flavour eigentstates are not identical to mass eigenstates.
\\ 
Subsequent experiments, such as the Sudbury Neutrino Observatory (SNO) in Canada and the KamLAND experiment in Japan, further confirmed neutrino oscillations by studying solar and reactor neutrinos, respectively~\cite{thomson2013modern}. 
Both Super-Kamiokande and SNO measured the flux of neutrinos originating from the Sun. 
They observed a significant deficit of electron-neutrinos compared to predictions from the Standard Solar Model (SSM), while the total flux of all neutrino flavours agreed with SSM calculations. 
\\
This procieded clear evidence that electron neutrinos produced through fusion in the Sun were oscillation into muon- and tau-neutrinos~\cite{berger2025particle}\cite{demtroder2016experimental}\cite{thomson2013modern}\cite{Zuber2016discovery}.
Neutrino oscillation thus demonstrates that lepton flavour is not conserved in the neutral lepton sector.
